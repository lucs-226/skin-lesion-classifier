\section{Experiments and Results}

\subsection{Experimental Setup}
The model was trained using PyTorch on a GPU-accelerated environment.
\begin{itemize}
    \item \textbf{Optimizer:} Adam optimizer with an initial learning rate of $1e-4$.
    \item \textbf{Scheduler:} `ReduceLROnPlateau` decays the learning rate by a factor of 0.5 if validation loss stagnates for 2 epochs.
    \item \textbf{Early Stopping:} Training halts if validation metric does not improve for 5 consecutive epochs.
\end{itemize}

\subsection{Test-Time Augmentation (TTA)}
To further boost performance and robustness during inference, we employ Test-Time Augmentation (TTA). For every test image, we generate $N=5$ augmented versions (including horizontal flips and slight rotations). The final prediction is obtained by averaging the softmax probabilities of these augmented views:
\begin{equation}
  \hat{y}_{final} = \frac{1}{N} \sum_{k=1}^{N} f(T_k(x))
\end{equation}
This ensemble-like approach effectively mitigates the variance of single-shot predictions and corrects misclassifications caused by ambiguous viewpoints.

\subsection{Results and Discussion}
The combination of WeightedRandomSampler and Mixup proved critical. The sampler ensured stable convergence for minority classes, while Mixup prevented the "memorization" of training samples.

\vspace{5pt}
\noindent\textbf{Metrics:} We report a Recall (Sensitivity) of approx. 0.85 for Melanoma on the external validation set. This high sensitivity confirms that the TTA and balanced sampling strategies successfully prioritized the identification of malignant lesions, fulfilling the primary clinical objective of minimizing false negatives. The confusion matrix shows a distributed error pattern, significantly reducing the bias towards the 'Nevus' class compared to a standard baseline.

\section{Conclusion}
We presented a dermoscopic classifier built on EfficientNet-B3. By integrating advanced sampling (WeightedRandomSampler), regularization (Mixup, Random Erasing), and inference techniques (TTA), we achieved a robust model capable of generalization on unseen data.
